\begin{table}
% Memory consumption is based on using only 11 element arrays. Current implementation is lazy and just uses a 16 byte array, but we never use the first two and last three, since they're not used for stack caching.
\centering
\caption{VM size and memory consumption}
\label{tbl-code-size-and-memory-consumption}
    \begin{threeparttable}
    \begin{tabular}{lrrrrrr} % UPDATED 20180327
    \toprule
                              & Size vs     & Size vs  &                      & AOT code  &   Break & Memory    \\
                              & interpreter & baseline &                      & reduction &   even  & usage     \\
    \midrule
    \midrule
    Baseline                  &     6863 B  &          &                      &           &         & 25 B      \\
    Improved peephole         &     7139 B  &   276 B  & \scriptsize   (+276) &  -14.6\%  &  1.9 KB & 25 B      \\
    Stack caching             &     7961 B  &  1098 B  & \scriptsize   (+822) &  -27.8\%  &  3.9 KB & 36 B      \\
    Popped value caching      &     9229 B  &  2366 B  & \scriptsize  (+1268) &  -33.1\%  &  7.1 KB & 80 B      \\
    Markloop                  &    12511 B  &  5648 B  & \scriptsize  (+3282) &  -33.4\%  & 16.9 KB & 87 B      \\
    Const shift               &    12955 B  &  6092 B  & \scriptsize   (+444) &  -34.3\%  & 17.8 KB & 87 B      \\
    16-bit array index        &    12935 B  &  6072 B  & \scriptsize    (-20) &  -38.7\%  & 15.7 KB & 87 B      \\
    SIMUL                     &    13001 B  &  6138 B  & \scriptsize    (+66) &  -39.2\%  & 15.7 KB & 87 B      \\
    Lightweight methods       &    13549 B  &  6686 B  & \scriptsize   (+548) &  -39.7\%  & 16.8 KB & 87 B      \\
    \bottomrule
    \end{tabular}
    \begin{tablenotes}
        \item The constant shift optimisation adds 170 bytes to the VM size. Because the MoteTrack benchmark cannot run without it, we cannot calculate the average code size reduction.
    \end{tablenotes}
    \end{threeparttable}
\end{table}
