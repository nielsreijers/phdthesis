\section{Manually optimising the Java source code}
\label{sec-optimisations-manual-java-source-optimisation}
As shown in Section \ref{sec-compilation-process}, our current implementation uses three steps to translate Java source code to Darjeeling bytecode: the standard Java compiler, the ProGuard optimiser, and Darjeeling's infuser. None of these do any complex optimisations. 

In a future version, ProGuard and the infuser should be merged into an 'optimising infuser' which uses normal, well-known optimisation techniques to produce better quality bytecode, but at the moment we do not have the resources to build such an optimising infuser.

Since our goal is to find out what level of performance is possible on a sensor node, we manually optimise the Java source to get better quality JVM bytecode from \mycode{javac}. While these changes are not an automatic optimisation we developed, we find it imporant to mention them explicitly and analyse their impact, since many developers may expect many of these to happen automatically, and without this step it would be impossible to reproduce our results.

We have been careful to limit ourselves to 'fair' optimisations, by which we mean optimisations that an optimising infuser could reasonably be expected to do automatically, given some basic, conservative assumptions about the performance model. 

The most common optimisations we performed are:
\begin{itemize}
	\item Store the result of expressions calculated in a loop in a temporary variable, if it is known the result will be the same for each iteration
	\item Since array and object field access is relatively expensive and not cached by the mark loop optimisation discussed in Section \ref{sec-optimisation-markloops}, prefer to store a value in a local variable if it may be used again soon rather than accessing the array or object twice
	\item Manually inlining small methods
	\item Prefer to use 16-bit variables for array indexes where possible
	\item Use bit shifts for multiplications by a power of two
	% TODO should we say more about 16-bit array indexes here? Actually this is not something an optimiser can solve. The reason the change has little effect on the C version is that the C version does less address calculations.
\end{itemize}

We will briefly examine the effect of some 'unfair' optimisations on the CoreMark benchmark in Section \ref{sec-evaluation-coremark}.


\paragraph{Temporary variables}
The first two optimisations generate extra store instructions, which means they may not always be beneficial if the value is never used again. But a value often only needs to be reused only once for it to be faster to store in a local variable than to calculate it twice. If we use the mark loops optimisation discussed in Section \ref{sec-optimisation-markloops}, in many cases the variable may be stored in registers, making accessing them either very cheap, or free.

\paragraph{Manual inlining}
We manually inline all small methods that were either a \mycode{\#define} in the C version of our benchmarks, or a function that was inlined by \mycode{avr-gcc}. ProGuard can also inline small methods, but when it does, it simply replaces the \mycode{INVOKE} instruction with the callee's body, prepended with \mycode{STORE} instructions to pop the parameters off the stack and initialise the callee's local variables. Manual inlining often results in better code, because it may not be necessary to store the parameters if they are only used once. Again, it is easy to imagine that an optimising compiler should be able to come to the same result automaticallly.

\paragraph{Platform independence}
Assuming an optimising infuser does raise the question how platform independent the resulting code is. If the infuser has more specific knowledge about the target platform, it can produce better code for that platform, but, while it should still run anywhere, this may not be as efficient on other platforms.

However, the optimisations described here are only based on very conservative assumptions that would work well for most devices in this class.

\paragraph{Example} An example of these manual optimisations, applied to the bubble sort benchmark, can be seen in Listing \ref{lst-manual-optimisation}. To have a fair comparison, we applied exactly the same optimisations to the C versions of our benchmarks, but here this had little or no effect on the performance.

\begin{listing}[H]
 \centering
 \begin{minipage}[t]{0.45\textwidth}
  \centering
  \begin{minted}[fontsize=\scriptsize]{java}
// ORIGINAL
public static void bsort(int[] numbers) {
  short NUMNUMBERS=(short)numbers.length;
  for (short i=0; i<NUMNUMBERS; i++) {
    for (short j=0; j<NUMNUMBERS-i-1; j++) {
      if (numbers[j]>numbers[j+1]) {
        int temp = numbers[j];
        numbers[j] = numbers[j+1];
        numbers[j+1] = temp;
      }
    }
  }
}
  \end{minted}
 \end{minipage}\hfill
 \begin{minipage}[t]{0.45\textwidth}
  \centering
  \begin{minted}[fontsize=\scriptsize]{java}
// MANUALLY OPTIMISED
public static void bsort(int[] numbers) {
  short NUMNUMBERS=(short)numbers.length;
  for (short i=0; i<NUMNUMBERS; i++) {
    short x=(short)(NUMNUMBERS-i-1);
    short j_plus_one = 1;
    for (short j=0; j<x; j++) {
      int val_at_j = numbers[j];
      int val_at_j_plus_one = numbers[j_plus_one];
      if (val_at_j>val_at_j_plus_one) {
        numbers[j] = val_at_j_plus_one;
        numbers[j_plus_one] = val_at_j;
      }
      j_plus_one++;
    }
  }
}
  \end{minted}
 \end{minipage}
\caption{Optimisation of the bubble sort benchmark}
\label{lst-manual-optimisation}
\end{listing}


