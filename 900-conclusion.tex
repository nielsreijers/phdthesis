\chapter{Conclusion}
This dissertation described the CapeVM sensor node virtual machine. CapeVM extends the state of the art by combining the desirable features of platform independent reprogramming, a safe execution environment, and acceptable performance. CapeVM was evaluated using a set of benchmarks, including small benchmarks to highlight specific behaviours, and five examples of real sensor node applications.

To come back to the research questions stated in Section \ref{sec-introduction-research-questions}, we can conclude the following:

\begin{enumerate}
	\item[a.]
	After identifying the sources of overhead in previous work on ahead-of-time compilers for sensor nodes, we introduced a number of optimisations to reduce the performance overhead by over 79\%, and the code size overhead by 60\%, resulting in an average performance overhead of 70\%, and the code size overhead of 79\% compared to native C.

	The optimisations introduced by CapeVM do increase the size of the VM, but the break-even point at which this is compensated for by the smaller code it generates, is well within the range of programme memory typically available on a sensor node.

	The price to pay for platform independence and a safe execution environment comes in three forms. There is still a performance overhead, but it is at a level that will be acceptable for many applications. The increase in code size however, and the space taken by the VM, do limit the size of applications we can load onto a device. Finally, the overhead in memory usage is a problem for programmes allocating many small objects.

	\item[b.]
	CapeVM's second contribution is providing a safe execution environment. Compared to native binary code, the higher level of abstraction of CapeVM's bytecode allowed us to develop a relatively simple set of safety checks.
	
	This results in a modest overhead in terms of VM size, and because most checks are performed at translation time, the overhead for providing safety is limited to a slowdown of 22\% and a 2\% increase in code size, compared to the unsafe version.
	
	Since to the best of our knowledge only two native code approaches exist that provide safety independent of the host, we cannot exclude the possibility that these could be further optimised. Currently however, CapeVM is on-par with or faster than existing systems, and provides platform independence at the same time.

	\item[c.]
	Regarding the question of whether Java is a suitable language for a sensor node VM, we can conclude some aspects of it are a good match. An advantage of its simple stack-based instruction set is that it can be implemented in a small VM, and while we showed the stack-based architecture introduces significant overhead, most of this overhead is eliminated by our optimisations.
	
	However, our benchmarks also exposed several problems that ultimately make standard Java a poor choice. Specifically, the lack of support for constant data, and the inefficient use of memory for programmes containing many small objects meant some benchmarks could not be ported directly from C to Java. We proposed several improvements, and developed an extension to allow constant data to be put in flash memory, but conclude that more work is necessary to come to a more sensor node specific language and make sensor node VMs truly useful in a wide range of real-world projects.
\end{enumerate}

\begin{table}
\caption{Comparison of CapeVM to related work}
\label{tbl-contribution-comparison}
    \begin{threeparttable}
    \begin{tabular}{lrrrrrrrr} % UPDATED 20180327
    \toprule
    Approach          & Platform    & Safe               & Performance           & Code size                  \\
                      & indep.      &                    &                       &                            \\
    \midrule
    \midrule
    Native code       & No          & No                 & 1x                    & 1x                         \\
    Interpreters      & Yes         & Mostly no          & 300-55400\% slower    & ~50\% smaller \tnote{b}    \\ 
    % Most interpreters & Yes         & No                 & 1300-22900\% slower   & ~50\% smaller \tnote{b}    \\
    % SensorScheme      & Yes         & Yes                & 300-10400\% slower    & unknown \tnote{b}          \\
    Ellul's AOT       & Yes         & No                 & 204-1927\% slower     & 26-449\% larger \tnote{b}  \\
    Safe TinyOS       & No          & Yes \tnote{a}      & 17\% slower           & 27\% larger                \\
    Harbor            & No          & Yes                & 380 to 1230\% slower  & 30 to 65\% larger          \\
    \emph{t-kernel}   & No          & Yes                & 50 to 200\% slower    & 500 to 750\% larger        \\
    CapeVM (unsafe)   & Yes         & No                 & 67\% slower           & 77\% larger                \\ % avg
    CapeVM (safe)     & Yes         & Yes                & 105\% slower          & 82\% larger                \\ % avg
    \bottomrule
    \end{tabular}
    \begin{tablenotes}
        \item[a] requires a trusted host
        \item[b] no support for constant data
    \end{tablenotes}
    \end{threeparttable}
\end{table}


Finally, we conclude by comparing our approach to existing work on improving sensor node VM performance, and on safe execution environments in Table \ref{tbl-contribution-comparison}.

Taking unsafe and platform specific native C as a baseline, we first note that existing interpreting sensor node VM's are typically not safe, and suffer from a 1 to 2 orders of magnitude slowdown. The performance overhead was reduced drastically by Ellul's work on Ahead-of-Time compilation, but still a significant overhead remains and this approach increases code size, reducing the size of programmes we can load onto a device.

On the safety side, Safe TinyOS achieves safety with relatively little overhead, but this depends on a trusted host. Harbor and \emph{T-kernel} provide safety independent of the host, but at the cost of a significant performance overhead, or increase in code size respectively. Non of these approaches provide platform independence.

Finally, we see CapeVM provides both platform independence and safety, at a cost in terms of code size and performance that is lower than or comparable to previous work.

%TODO: discuss with prof. Shih whether to include SensorScheme in Table 9.1