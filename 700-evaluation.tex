\chapter{Evaluation}
\label{sec-evaluation}
%TODO EVALUATION: explain why motetrack is so slow. one is the deep object structure, another seems to be reading the signatures, which can't be done by simple mem copy as in the native version
%TODO EVALUATION: add more detail on individual benchmarks
%TODO EVALUATION: add some worst case benchmarks
%TODO EVALUATION: add section to investigate how the VM may perform on other platforms. vary number if registers available, and compare 32/16/8 bit versions of some benchmarks
%TODO EVALUATION: add conclusion to discuss variation in results. YMMV.

This dissertation presents a number of techniques to make sensor node virtual machines safe, and improve their performance, staying within the constraints set out in Section \ref{sec-myvm-goals}. This chapter evaluates how MyVM meets these goals by measuring its performance and code size overhead for a number of different benchmarks.

First, Section \ref{sec-evaluation-benchmarks} will describe our experimental setup, the benchmarks used, and how the source code for these benchmarks was obtained.

Next, Section \ref{sec-evaluation-coremark} will use our largest benchmark to examine the effect of the lack of optimisations done by the standard \mycode{javac} compiler, and the manual optimisations we do on the Java source that in a future version should be done by an optimising compiler/infuser.

Sections \ref{sec-evaluation-aot-translation-performance} and \ref{sec-evaluation-aot-translation-code-size} evaluate the result of our optimisations to the AOT translation process on performance and code size respectively.

Section \ref{sec-evaluation-method-invocation} evaluates the cost of method invocation and the effect of our lightweight method calls.

The cost of adding safety checks is examined in Section \ref{sec-evaluation-safety}.

Platform independence is one of the main reasons to use a VM. While we only implement our approach for the Atmel AVR platform, Section \ref{sec-evaluation-other-platforms} presents measurements that give an indication of the expected performance on other sensor node platforms.

In Section \ref{sec-evaluation-limitations} we discuss the limitations and cost of using a VM, and describe some known hard cases our VM currently does not handle as well.

Finally we conclude this chapter in Section \ref{sec-evaluation-conclusion} by comparing our results to existing approaches to improve sensor node VM performance or provide a safe execution environment.



