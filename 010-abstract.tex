\begin{abstractzh}
中文摘要

\bigbreak
\noindent \textbf{關鍵字:}{\, \makeatletter \@keywordszh \makeatother}
\end{abstractzh}

\begin{abstracten}
Many virtual machines have been developed targeting resource-constrained sensor nodes. While packing an impressive set of features into a very limited space, most fall short in two key aspects: performance, and a safe, sandboxed execution environment. Since most existing VMs are interpreters, a slowdown of one to two orders of magnitude is common. Given the limited resources available, verification of the bytecode is typically omitted, leaving them vulnerable to a wide range of possible attacks.

In this dissertation we propose CapeVM, a sensor node VM aimed at delivering both high performance and a sandboxed execution environment that guarantees malicious code cannot corrupt the VM's internal state or perform actions not allowed by the VM.

CapeVM uses Ahead-of-Time compilation to native code to improve performance and introduces a range of optimisations to eliminate most of the overhead present in previous work on sensor node AOT compilers. A safe execution environment is guaranteed by a set of run-time and translation-time checks. The simplicity of the VM's instruction set allows us to perform most of these checks when the bytecode is translated to native code, reducing the need for expensive run-time checks compared to native code approaches.

We evaluate CapeVM using a set of 12 benchmarks with varying characteristic, including the commercial \mybench{CoreMark} benchmark and a number of real sensor node applications. While some overhead from using a VM and added safety checks cannot be avoided, the evaluation shows CapeVM's optimisations reduce this overhead dramatically. This results in a performance 2.1x slower than unsafe native code, which is comparable to or better than existing native solutions to provide safety. Without safety checks, the overhead drops to 1.7x. Thus, CapeVM combines the desirable properties of existing work on both safety and virtual machines for sensor networks with significantly improved performance.
\bigbreak
\noindent \textbf{Keywords: }{\, \makeatletter \@keywordsen \makeatother}
\end{abstracten}

\begin{comment}
% \category{I2.10}{Computing Methodologies}{Artificial Intelligence --
% Vision and Scene Understanding} \category{H5.3}{Information
% Systems}{Information Interfaces and Presentation (HCI) -- Web-based
% Interaction.}

% \terms{Design, Human factors, Performance.}

% \keywords{Region of interest, Visual attention model, Web-based
% games, Benchmarks.}
\end{comment}
