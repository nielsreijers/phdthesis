\begin{abstractzh}
中文摘要

\bigbreak
\noindent \textbf{關鍵字:}{\, \makeatletter \@keywordszh \makeatother}
\end{abstractzh}

\begin{abstracten}
Many virtual machines have been developed targeting resource-constrained sensor nodes. While packing an impressive set of features into a very limited space, most fall short in two key aspects: performance, and a safe, sandboxed execution environment. Since most existing VMs are interpreters, a slowdown of one to two orders of magnitude is common. Given the limited resources available, verification of the bytecode is typically omitted, leaving them vulnerable to a wide range of possible attacks.

In this paper we propose MyVM, a sensor node JVM based on the Darjeeling VM, and aimed at delivering both high performance and a sandboxed execution environment that guarantees malicious code cannot corrupt the VM's internal state or perform actions not allowed by the VM.

MyVM uses Ahead-of-Time compilation to native code to improve performance and introduces a range of optimisations to eliminate most of the overhead still present in previous work on sensor node AOT compilers. Safety is guaranteed by a set of run-time and translation-time checks. The simplicity of the JVM's instruction set allows us to perform most of these checks when the bytecode is translated to native code, reducing the need for expensive run-time checks.

Using a set of 12 benchmarks with varying characteristic, including the standard CoreMark benchmark, and a number of real sensor node applications, we show this results in an average performance roughly 2.1x slower than unsafe native code. Without safety checks, this drops to 1.7x. Thus, MyVM combines the desirable properties of existing work on both safety and virtual machines for sensor networks, while delivering performance and code size overhead comparable or better than existing solutions.
\bigbreak
\noindent \textbf{Keywords: wireless sensor networks, Internet of Things, Java, virtual machines, ahead-of-time compilation, software fault isolation}{\, \makeatletter \@keywordsen \makeatother}
\end{abstracten}

\begin{comment}
% \category{I2.10}{Computing Methodologies}{Artificial Intelligence --
% Vision and Scene Understanding} \category{H5.3}{Information
% Systems}{Information Interfaces and Presentation (HCI) -- Web-based
% Interaction.}

% \terms{Design, Human factors, Performance.}

% \keywords{Region of interest, Visual attention model, Web-based
% games, Benchmarks.}
\end{comment}
