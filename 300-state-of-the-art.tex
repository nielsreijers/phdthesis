\chapter{State of the art}
This chapter presents the state of the art in Internet of Things and sensor networks relevant to the work in this thesis. It starts with existing work on programming WSN and IoT networks, and the virtual machines developed for them. Next, it discusses proposed ways to improve sensor node VM performance, and ways to guarantee safety on sensor devices.

\section{Programming Internet-of-Things devices}

This challenge of programming IoT devices can be split into two questions:

\begin{itemize}
	\item How can we build applications at a higher level, coordinating the behaviour of many devices without having to specify the behaviour from each device's individual perspective?
	\item How can we best reprogramme individual nodes safey and efficiently to support these applications?
\end{itemize}

This thesis is concerned with the second question, and argues that a virtual machine can be an attractive option in many scenarios. But we will first discuss the higher level question of how to programme WSN/IoT applications and use one of these systems as a motivating example.

\subsection{Programming IoT applications}



Cornell’s MagnetOS \cite{Liu:2005wsa}, is interesting because it proposes a novel programming model which allows the user to write the application as a single Java application, not explicitly related to individual nodes. The system then automatically partitions the applications into pieces (by default along object boundaries), and places these pieces on nodes in the network in such a way as to minimise energy consumption. However, it requires nodes significantly more powerful than what we expect to find in a typical WSN. We consider this to be a possible future direction for our study if and when all future WSNs are proved to be more powerful.

In Agilla \cite{Fok:2005bh} programs consist of software agents that can move around autonomously in the network. While this allows some behaviours to be expressed in a natural way, the paradigm is very different from conventional languages, and the assembly-like instruction set which is based on the Maté VM \cite{Levis:2002ku} discussed below makes it hard to use.

ADAE \cite{Chang:2010ek}, developed at the IT University of Copenhagen, configures the network according to a policy describing the desired data quality, including fallback options which the system may use in case the ideal situation cannot be achieved. ADAE then dynamically reconfigures the network in response to changing conditions such as node failures, unexpected power drops, or interesting events detected by the network. However, the language used to describe the policy and constraints is hard to use and it’s likely a skilled engineer is still needed to translate the user’s requirements into the constraint optimisation problem ADAE uses as input.


LooCI \cite{Hughes:dg} is a component infrastructure middleware for WSNs with standard support for runtime reconfiguration. The LooCI component model supports dynamic binding and interoperability between different hardware platforms, in their case a typical sensor node and the slightly more powerful Sun SPOT, and different languages, with implementations in C and Java. However, it considers component selection to be higher-level services outside of its core functionality.
%TODO: read LooCI paper again.
%TODO: use from LooCI somewhere: it uses Java and C
%TODO: use from LooCI somewhere: it has a nice list of requirements for middleware, including it shouldn't use too much memory and should be fast.

-----------------------------------wip







-----------------------------------old text

There have been several projects which have developed virtual machines for very resource constrained devices in general, or for sensor networks specifically. Designed specifically for WSN, UC Berkeley’s Maté [2] was one of the first to prove VMs can be made to run on sensor nodes. Recently ETH Zürich has developed a more powerful VM called SwissQM [3]. It is based on a small subset of the JVM instruction set, extended with several sensor network specific instructions. In both systems however, the application has to be programmed in very low level, assembly-like language.
Several VMs for more high level languages like Python or Java have also been developed, which fit on severely resource constrained nodes. Almost all rewrite the original bytecode to their own format and employ various techniques to reduce code size. Some functionality is sacrificed in order to fit on the sensor nodes, for instance reflection is typically not supported.
The Python-on-a-chip project [4] developed a Python bytecode interpreter small enough to fit on sensor nodes. Requiring 55K of program memory and a recommended 8K of RAM, it fits on many, but not all of the current sensor boards. Darjeeling [5] from Delft University of Technology is able to run modified Java bytecode on sensor nodes, with only minor restrictions. It has no support for floating point, 64-bit datatypes, reflection or synchronised method calls, although the later is not really a restriction since synchronised blocks are supported. Similarly TakaTuka [6] is a slightly more complete and more complex JVM, also running on both AVR and MSP based sensor nodes and offering it’s own debugger. Sun’s Squawk VM [7] is significantly larger than Darjeeling and TakaTuka, but offers full CLDC [8] compliance.
The opposite approach is taken by NanoVM [9], which takes a minimal approach, trading support for a large number of java opcodes for simplicity and code size. The whole VM fits in the 8K flash of an ATMega8 CPU, leaving 512bytes of EEPROM and 75% of the CPU’s 1K RAM to the application. It’s small size and simple design make it a suitable choice to build our initial prototypes on.

While these technologies allow the application to be developed in a higher level language, and to be deployed on heterogeneous networks, applications still have to be written in an imperative way, and from the node’s perspective. These technologies will serve as a basis on which we build our middleware. We plan to enhance the VM design with the policy framework so that WSN applications can be coded in a very high level manner, similar to a script language for many new application domains.
Very different approaches are taken by Agilla [10] and MagnetOS [11]. 

On profile related research, sMap [12] provides an attractive and flexible RESTful interface through which we can discover what sensors are available at a certain node and get or set several configuration properties. Although the authors succeeded in dramatically reducing the footprint, it is still relatively resource intensive. It is also limited in the number of properties it exposes, but the idea could easily be expanded to include more diverse node types and configuration options. ZigBee’s approach [13] is less resource intensive, and offers a more complete description of a node and can be applied to more devices. However it is less flexible since it works with a fixed set of device descriptors. In the Z-Wave community, a device class is used to define device profiles. Each profile contains a set of required and recommended command classes. Z-Wave defines a device into a profile which lists the device class and all supported command classes. The profiles are quite generic and fixed.



There are projects that have addressed some parts of our goals. 



There have been several projects which have developed virtual machines for very resource constrained devices, either using general purpose languages like Python or Java (Brouwers, Langendoen & Corke, 2009, Aslam, et al., 2008, Harbaum), or specialised for WSN (Levis & Culler, 2002, Müller, Alonso and Kossmann, 2007). These technologies allow the application to be developed in a higher level language on heterogeneous networks. However, applications are still written in an imperative way, and from the node’s perspective.
Very different approaches are taken by Agilla (Fok, Roman & Lu, 2005) and MagnetOS (Liu, et al., 2005)



% todo: mention s-map and zigbee profiles (take text from lin et al)


\subsection{WuKong}
Motivating example


\section{Sensor node virtual machines}
Many VMs have been proposed that are small enough to fit on a resource-constrained sensor node. They can be divided into two categories: generic VMs and application-specific VMs, or ASVMs \cite{Culler05} that provide specialised instructions for a specific problem domain. One of the first VMs proposed for sensor networks, Mat\'e \cite{Levis:2002ku}, is an ASVM. It provides single instructions for tasks that are common on a sensor node, so programmes can be very short. Unfortunately they have to be written in a low-level assembly-like language, limiting its target audience. SwissQM \cite{Muller:2007fs} is a more traditional VM, based on a subset of the Java VM, but extended with instructions to access sensors and do data aggregation. VM* \cite{Koshy:2005ww} sits halfway between the generic and ASVM approach. It is a Java VM that can be extended with new features according to application requirements. Unfortunately, it is closed source.

Several generic VMs have also been developed, allowing the programmer to use general purpose languages like Java, Python, or even LISP \cite{Harbaum, Brouwers:2009cj, Aslam:2008, Evers:2010ur}. The smallest official Java standard is the Connected Device Limited Configuration \cite{CLDC}, but since it targets devices with at least a 16 or 32-bit CPU and 160-512KB of flash memory available, it is still too large for most sensor nodes. The available Java VMs for sensor nodes all offer some subset of the standard Java functionality, occupying different points in the tradeoff between the features they provide, and the resources they require.



\section{Performance}
Only a few papers describing sensor node VMs contain detailed performance measurements. TinyVM \cite{Hong:2009gc} reports a slowdown between 14x and 72x compared to native C, for a set of 9 benchmarks. DVM \cite{Balani:2006} has different versions of the same benchmark, where the fully interpreted version is 108x slower than the fully native version. Ellul reports measurements on the TakaTuka VM \cite{Aslam:2008, Ellul:2012thesis} where the VM is 230x slower than native code, and consumes 150x as much energy. SensorScheme \cite{Evers:2010ur} is up to 105x slower. Finally, Darjeeling \cite{Brouwers:2009cj} reports between 30x and 113x slowdown. Since performance depends on many factors, it is hard to compare these numbers directly. But the general picture is clear: current interpreters are one to two orders of magnitude slower than native code.

Translating bytecode to native code to improve performance has been a common practice for many years. A wide body of work exists exploring various approaches, either offline, ahead-of-time  or just-in-time. One common offline method is to first translate the Java code to C as an intermediate language, and take advantage of the high quality C compilers available \cite{Muller:1997}. Courbot et al. describe a different approach, where code size is reduced by partly running the application before it is loaded onto the node, allowing them to eliminate code that is only needed during initialisation \cite{Courbot:2010}. Although the initialised objects are translated to C structures that are compiled and linked into a single image, the bytecode is still interpreted. While in general we can produce higher quality code when compiling offline, doing so sacrifices key advantages of using a VM.

Hsieh et al. describe an early ahead-of-time compiling desktop Java VM \cite{Hsieh:1996cy}, focussing on translating the JVM's stack-based architecture to a register based one. In the Japale\~no VM, Alpern et al. take an approach that holds somewhere between AOT and JIT compilation \cite{Alpern:1999}. The VM compiles all code to native code before execution, but can choose from two different compilers to do so. A fast baseline compiler simply mimics the Java stack, but either before or during run-time, a slower optimising compiler may be used to speed up critical methods.

Since JIT compilers work at run-time, much effort has gone into making the compilation process as light weight as possible, for example \cite{Krall:1998}. More recently these efforts have included JIT compilers targeted specifically at embedded devices. Swift \cite{Zhang:2012wf} is a light-weight JVM that improves performance by translating a register-based bytecode to native code. But while the Android devices targeted by Swift may be considered embedded devices, they are still quite powerful and the transformations Swift does are too complex for the ATmega class of devices. HotPathVM \cite{Gal:2006} has lower requirements, but at 150KB for both code and data, this is still an order of magnitude above our target devices.

Given our extreme size constraints - ideally we only want to use in the order of 100 bytes of RAM to allow our approach to be useful on a broad range of devices, and leave ample space for other tasks on the device - almost all AOT and JIT techniques found in literature require too much resources. Indeed, some authors suggest sensor nodes are too restricted to make AOT or JIT compilation feasible \cite{Aslam:2011thesis, Wirjawan:2008}.


% NOTE: the 811\% here comes from the manual optimisation table. since we calculate overhead slightly different there to be able to split it into 3 components, the total in that table is 811 instead of 815 in the trace output text file.
On the desktop, VM performance has been studied extensively, but for sensor node VMs this aspect has been mostly ignored. To the best of our knowledge AOT compilation on a sensor node has only been tried by Ellul and Martinez \cite{Ellul:2010iw}, and our work builds on their approach. They improve performance considerably compared to the interpreters, but there is still much room for improvement. Using the standard CoreMark benchmark, their approach generates code that is 811\% slower and 245\% larger than optimised native C. While the reduced throughput may be acceptable for some applications, there are two other reasons why it is important to improve on these results: the loss of performance results in an equivalent increase in cpu power consumption, thus reducing battery life. More importantly, the increased size of the compiled code reduces the amount of code we can load onto a node. Given that flash memory is already restricted, this is a major sacrifice to make when adopting AOT on sensor nodes.

\subsection{Effect on energy consumption}

\section{Safety}
\label{sec-state-of-the-art-safety}
With some exceptions \cite{Evers:2010ur}, most current sensor node VMs don't discuss safety, but instead focus on the functionality provided and how this can be implemented on a tiny sensor node. This is unfortunate, because the ability to provide a safety execution environment is both desirable, and easier to implement using a VM than it is using native code.

Several non-VM systems exists to provide various levels of safety for sensor nodes. They fall into two distinct categories: they either depend on a trusted host, or allow the node to guarantee safety independent of whatever malicious code it may receive from the host. Obviously the latter is the stronger guarantee, but it also comes at a higher price.

\subsection{Source code approaches}
A number of systems have been developed that guarantee safety at the source code level. Virgil \cite{Titzer:2006uy} is a language that is inherently safe and specifically designed for sensor nodes. The application is explicitly split into an initialisation and run-time phase, where objects are only allocated during the initialisation phase. The initialisation phase happens during compilation (to C code), which means all object and their locations are know at this point, allowing Virgil to ensure safety and optimise the code at compile time.

Safe TinyOS on the other hand, works on annotated nesC TinyOS code. It uses the Deputy \cite{Condit:2007uo} source-to-source compiler to analyse the C source code and insert the necessary run-time checks were necessary. Because this happens on the host, before sending the code to the node, it can use the host's resources to do more complex analysis of the source code and eliminate checks where the it can determine a memory access to be safe at compile time, resulting in a much lower overhead

Both approaches eventually result in standard C, which is then compiled and sent to the node. Therefore, both approaches may protect against accidental programming errors, but do not protect the node from malicious code an attacker may send to it.

\subsection{Native code approaches}
For desktop applications, Wahbe et al. described software fault isolation \cite{Wahbe:1994cj} techniques to isolate a piece of untrusted code, without the overhead of using processes and the CPU's memory protection. A typical example is a plugin that frequently needs to interact with an application. It should be isolated from the application so bugs in the plugin can't bring down the whole application, but running it as a separate process would incur high overhead due to frequent context switches. Two basic methods are described to isolate such code from the main application: we can either rewrite the native code at load time, inserting checks at all potentially unsafe writes, or we can compile the code to a more restricted format with the appropriate checks already in place, and verify the code adheres to this standard at load time.

Since we don't have processes or CPU memory protection on a sensor node, Wahbe's approach provides an attractive alternative. Two systems exist that provide safety for sensor nodes using each of these approaches. \emph{t-kernel} raises the level of system abstraction for the developer by providing three features typically missing on sensor nodes: preemptive scheduling, virtual memory, and memory protection. It does this by extensive rewriting of the binary code at load time. While \emph{t-kernel} is heavily optimised, the price for this is that programmes still run 50-200\% slower, and code size increases by 500-750\%.

The other approach is taken by Harbor \cite{Kumar:2007ge}, which consists of two components. On the desktop a binary rewriter sandboxes an application by inserting run-time checks before it is sent to the node. The SOS operating system \cite{Han:2005th} is then extended with a binary verifier to verify incoming binaries. The correctness only depends on the correctness of this verifier. The increase in code size is more modest than for \emph{t-kernel} at a 30-65\% increase. But performance is 160-1230\% slower, where the authors note the benchmark producing the highest slowdown is more typical of sensor node code. They also note more complex analysis of the binary code could reduce the number of necessary checks, but this would significantly increase the complexity of the verifier.

