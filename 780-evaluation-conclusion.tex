\section{Conclusion}
\begin{table*}[]
\centering
\caption{Comparison of our approach to related work}
\label{tbl-contribution-comparison}
\begin{tabular}{lrrrrrrrr}
\toprule
Approach        & Platform    & Safe               & Performance        & Code size               \\
                & independent &                    &                    & \\
\midrule
Native code     & No          & No                 & 1x                 & 1x                      \\
Interpreters    & Yes         & Mostly no          & 300-23000\% slower & ~50\% smaller           \\
Ellul's AOT     & Yes         & No                 & 123-844\% slower   & 120-346\% larger        \\
Safe TinyOS     & No          & Depends on a trusted host   & 17\% slower        & 27\% larger             \\
\emph{t-kernel} & No          & Yes                & 50-200\% slower    & 500-750\% larger        \\
Harbor          & No          & Yes                & 160-1230\% slower  & 30-65\% larger          \\
Our VM (unsafe) & Yes         & No                 & 20-110\% slower    & 17-120\% larger         \\
Our VM (safe)   & Yes         & Yes                & 22-185\% slower    & 20-124\% larger         \\
\bottomrule
\end{tabular}
\end{table*}


Since we consider both safety and platform independence to be desirable properties for sensor networks, we conclude our evaluation by comparing our approach to existing work on both sensor node virtual machines and safety in Table \ref{tbl-contribution-comparison}.

Taking unsafe and platform specific native C as a baseline, we first note that existing interpreting sensor node VM's are typically not safe, and suffer from a 1 to 2 orders of magnitude slowdown. The performance overhead was reduced drastically by Ellul's work on Ahead-of-Time compilation, but still a significant overhead remains and this approach increases code size, reducing the size of programmes we can load onto a device.

On the safety side, Safe TinyOS achieves safety with relatively little overhead, but this depends on a trusted host. t-kernel and Harbor provide safety independent of the host, but at the cost of a significant increase in code size, or performance overhead respectively. Non of these approaches provide platform independence.

Finally, we see our VM provides both platform independence and safety, at a cost, both in terms of code size and performance, that is lower than or comparable to previous work.

Coming back to the question of whether a VM is a good way to provide security, we first note that since to the best of our knowledge only two such systems exists, we cannot exclude the possibility that native code approaches could be further optimised to achieve better performance.

However our results show that currently our approach is on-par with or faster than the two existing native code approaches, and provides platform independence at the same time. We also note that if we require a platform independent way of reprogramming our nodes, making it safe comes at a relatively low extra cost, with our safe VM only 18\% slower than the unsafe version.