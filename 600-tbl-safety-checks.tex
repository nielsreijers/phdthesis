\newcounter{tcheckcnt}
\newcommand{\tcheck}[1]{\refstepcounter{tcheckcnt}T-\arabic{tcheckcnt}\label{#1}}
\newcounter{rcheckcnt}
\newcommand{\rcheck}[1]{\refstepcounter{rcheckcnt}R-\arabic{rcheckcnt}\label{#1}}

\makeatletter
\renewcommand\p@tcheckcnt{T-\arabic{tcheckcnt}\expandafter\@gobble}
\renewcommand\p@rcheckcnt{R-\arabic{rcheckcnt}\expandafter\@gobble}
\makeatother

\begin{table*}
\centering
\caption{List of safety checks}
\label{tbl-safety-checks}
\begin{tabular}{lp{0.9\linewidth}}
\toprule
 & Translation-time checks \\

\tcheck{chk-method-header-is-sane}
	& For each method header, the number of own local variable slots <= the number of total variable slots, the number of (int/ref) arguments <= the number of (int/ref) variables, static methods are not abstract. \\

\tcheck{chk-return-or-goto-at-end-of-method}
	& The last instruction of each method is a \mycode{RETURN} or \mycode{GOTO}. \\

\tcheck{chk-brtarget-exists}
	& Branch instructions branch to an index < the number of \mycode{BRTARGET}s announced in the method header. \\

\tcheck{chk-all-brtargets-found}
	& At the end of each method, we have seen the exact number of \mycode{BRTARGET} instructions announced in the method header. \\

\tcheck{chk-invokelight-target-found}
	& The target for an \mycode{INVOKELIGHT} call is already translated, so the target address is known. \\

\tcheck{chk-invokestatic-target-header-found}
	& The target method header for an \mycode{INVOKESTATIC}/\mycode{INVOKESPECIAL} exists. \\

\tcheck{chk-stack-is-empty-after-return}
	& After popping the return value the stack is empty. \\

\tcheck{chk-sufficient-stack-space-at-invokelight}
	& At the point of an \mycode{INVOKELIGHT} instruction, the max stack of the caller >= the current stack depth - the number of arguments to the callee + the max stack of the callee. \\

\tcheck{chk-sufficient-locals-at-invokelight}
	& For each \mycode{INVOKELIGHT}, the total number of slots - the number of own variable slots for the caller >= the total variable slots for the callee. \\

\tcheck{chk-stack-is-empty-at-branches}
	& The stack is empty at branches and branch targets. \\

\tcheck{chk-no-operandstack-underflow}
	& Before each instruction, the stack depth >= the number of elements to be consumed by the instruction. \\

\tcheck{chk-no-operandstack-overflow}
	& After each instruction, the stack depth <= the max stack depth announced in the header. \\

\tcheck{chk-local-variable-slot-exists}
	& The index of the local variable < the number of own variable slots for the current method. \\

\tcheck{chk-static-variable-infusion-exists}
	& The target infusion of a static variable exists. \\

\tcheck{chk-static-variable-slot-exists}
	& The index of the static variable < the number of static variable slots for the target infusion. \\

\midrule
& Run-time checks \\

\rcheck{chk-invokevirtual-target-found}
	& The target implementation for an \mycode{INVOKEVIRTUAL}/\mycode{INVOKEINTERFACE} is found. \\

\rcheck{chk-no-nativestack-overflow}
	& Whenever a new stack frame is allocated the frame+max stack depth+some safety margin > the end of the heap. \\

\rcheck{chk-invokevirtual-stack-effects-match}
	& The target implementation for an \mycode{INVOKEVIRTUAL}/\mycode{INVOKEINTERFACE} matches the stack effects used to verify the caller’s stack at translation time. \\

\rcheck{chk-memory-access-within-heap}
	& The target address of an array element or object field is within the heap. \\

\bottomrule
\end{tabular}
\end{table*}