\begin{table}
\caption{Benchmarks used in the evaluation}
\label{tbl-benchmarks}
    \centerline{
    \begin{tabular}{lp{.40\textwidth}ccc}
    \toprule
    Benchmark                   & Source and input data                                                                                      & Size             & Typical sensor & Used as a                                \\
                                &                                                                                                            &                  & node code      & benchmark in                             \\
    \midrule
    \midrule
    \mybench{Bubble sort}       & Darjeeling sources \cite{darjeelingsource}                                                                 & single function  & no             & \cite{Brouwers:2009cj, Ellul:2012thesis} \\
                                & \emph{Input:} 256 16-bit numbers sorted in reverse order, as in the original source.                       &                  &                & \\
    \mybench{Heap sort}         & Standard heap sort taken from \cite{heapsort}                                                              & two functions    & no             & \\
                                & \emph{Input:} 256 16-bit numbers sorted in reverse order.                                                  &                  &                & \\
    \mybench{Binary search}     & TakaTuka sources \cite{takatukasource}                                                                     & single function  & no             & \cite{Ellul:2012thesis}                  \\
                                & \emph{Input:} worst case (not found) search in 100 16-bit values, as in \cite{Ellul:2012thesis}.           &                  &                & \\
    \mybench{XXTEA}             & Wheeler and Needham \cite{Wheeler:1998}                                                                    & single function  & no             & \\
                                & \emph{Input:} 32 32-bit numbers. Contents do not affect performance.                                       &                  &                & \\
    \mybench{MD5}               & Darjeeling sources \cite{darjeelingsource}                                                                 & single function  & no             & \cite{Brouwers:2009cj, Ellul:2012thesis} \\
                                & \emph{Input:} the string 'message digest' as in the original source.                                       &                  &                & \\
    \mybench{RC5}               & LibTomCrypt \cite{libtomcrypt}                                                                             & single function  & no             & \\
                                & \emph{Input:} first test case in libtomcrypt sources (64 bit)                                              &                  &                & \\
    \mybench{FFT}               & fixed point FFT using the widespread \mycode{fix\_fft.c} \cite{sos-operating-system}                       & single function  & yes            & \cite{Kumar:2007ge}                      \\
                                & \emph{Input:} 64 8-bit or 16-bit numbers. Contents do not affect performance.                              &                  &                & \\
    \mybench{Outlier detection} & our implementation of the algorithm described in \cite{Kumar:2007ge}                                       & single function  & yes            & \cite{Kumar:2007ge}                      \\
                                & \emph{Input:} 20 16-bit values increasing from 0 to 19, with outliers of 1000 and -1000 at index 2 and 11. &                  &                & \\
    \mybench{LEC}               & our implementation of the compression algorithm described in \cite{Marcelloni:2009ja}                      & single function  & yes            & \\
                                & \emph{Input:} 256 16-bit ECG measurements downloaded from PhysioNet \cite{physionet-ecg-data}.             &                  &                & \\
    \mybench{CoreMark 1.0}      & EEMBC \cite{coremark}                                                                                      & full application & no             & \\
                                & \emph{Input:} defined in CoreMark source.                                                                  &                  &                & \\
    \mybench{MoteTrack}         & Lorincz \cite{Lorincz:2006fc, motetrack}                                                                   & full application & yes            & \\
                                & \emph{Input:} defined in MoteTrack source.                                                                 &                  &                & \\
    \mybench{Heat detection}    & adapted from code used in our group to track objects using an 8x8 pixel heat sensor                        & full application & yes            & \\
                                & \emph{Input:} 101 frames of 8x8 16-bit values for calibration, 30 frames for detection.                    &                  &                & \\
    \bottomrule
    \end{tabular}
    }
\end{table}


\section{Benchmarks and experimental setup}
\label{sec-evaluation-benchmarks}

This section describes our experimental setup, the benchmarks used and how the source code for each benchmark was obtained, and any relevant details in their implementation.

A set of twelve different benchmarks, shown in Table \ref{tbl-benchmarks}, is used to measure the effect of the optimisations and the overhead of safety checks. This mix of benchmarks was chosen for several reasons. A number of benchmarks, \mybench{bubble sort}, \mybench{binary search}, \mybench{MD5}, \mybench{FFT}, and \mybench{Outlier detection} were chosen because they are used in various related work, allowing a comparison of CapeVM to these results.

A number of benchmarks are small benchmarks that process an array of data. Processing arrays of data is common to many sensor node application, and likely to be performance sensitive. While the actual processing done may not be typical for sensor network (although the \mybench{MoteTrack} application does do a bubble sort), the small size of these benchmarks make them useful to highlights specific behaviours that would be lost in the averages of a larger benchmark.

The \mybench{CoreMark} benchmark is an industry standard benchmark to measure CPU performance. It is a larger benchmark, mixing several kinds of processing: linked list processing, a state machine, and matrix operations. It contains many different methods, which enables us to evaluate the effect of method calls. Since it mixes different kinds of processing, it is a good example of the expected average behaviour.

Finally, \mybench{Outlier detection}, \mybench{LEC}, \mybench{MoteTrack} and \mybench{heat detection} are all code that was specifically developed for sensor nodes, and \mybench{FFT} is a typical signal processing operation, which is a common task for sensor nodes.


\subsection{Implementation details}
To ensure the results can be reproduced, we describe the implementation of our benchmarks in this section. For most benchmarks a C version is available in the sources mentioned in Table \ref{tbl-benchmarks}. The sources for \mybench{heat detection}, \mybench{LEC} and \mybench{outlier detection} are not available online, but are listed in the appendices.

The \mybench{bubble sort}, \mybench{heap sort}, \mybench{FFT}, \mybench{binary search}, and \mybench{outlier detection} benchmarks could all be implemented for different data sizes. In this evaluations 16-bit data is used. 8-bit data is too small for many tasks, for example the ATmega's has 10-bit ADCs, while the memory constraints of sensor nodes mean developers will often be reluctant to use 32-bit variables where 16 bits are sufficient. Therefore, the middle option is used for the main evaluation, and the effect on the performance of 8-bit or 32-bit data is discussed in Section \ref{sec-evaluation-other-platforms}.

The C versions of these benchmarks were translated directly to Java, keeping both implementations as close as possible, and manually optimised as described in Section \ref{sec-optimisations-manual-java-source-optimisation}. These optimisations did not affect the performance of the C version significantly, indicating \mycode{avr-gcc} already does similar transformations on the original code.

There are cases where a developer who is aware of the performance characteristics of the VM may choose a different approach than the one used in the C version when developing a Java implementation directly. We discuss some of the issues when translating C to Java for the \mybench{CoreMark} benchmark in Section \ref{sec-evaluation-coremark-non-automatic-optimisations}, including two choices that would have lead to better performance. The C version was followed as closely as possible to avoid bias by optimising specifically for our VM. We take a bit more liberty for the \mybench{MoteTrack} and \mybench{heat detection} benchmarks, since these could not be directly translated.

The benchmarks exposed some limitations of using a VM instead of native code, which are common to most sensor nodes VMs. Specifically, the lack of support for constant data, high memory overhead for code containing many small objects, and high performance overhead for allocating temporary objects. These are discussed in more detail in Chapter \ref{sec-lessons-from-jvm}, where we also suggest options to solve some of these limitations. In this section we describe the cases where they forced us to deviate from our preferred direct C to Java mapping.

\subsubsection{FFT}
Both 8-bit and 16-bit versions of the \mycode{fix_fft.c} implementation exist. In the main evaluation we use the 16-bit version taken from the Harbor source code \cite{sos-operating-system}.

Both versions contain a table of precalculated sine wave values, which are stored in flash using the constant array optimisation introduced in Section \ref{sec-opt-constant-arrays}.

\subsubsection{Outlier detection}
We implemented the outlier detection algorithm as described in \cite{Kumar:2007ge}:

\begin{displayquote}
"The outlier detector samples a set of sensor values and stores them in a buffer. Once the buffer is filled, it computes the distance between all pairs of samples in the buffer and stores the result in a matrix. Using a distance threshold, the algorithm marks the distance measurements in the matrix that are greater than the threshold. If the majority of the distance measurements for a sensor readings are marked, then the sensor reading is classified as an outlier."
\end{displayquote}

Note that there is no reason to store the distances in a matrix, and having to allocate this matrix limits us to only small arrays of input data. The same result can be calculated directly, without the distance matrix, by examining the samples one at a time, and counting the number of other samples with a distance higher than the threshold.

Because this benchmark will be used to compare CapeVM's safety cost to Harbor's, we implemented it as described in the paper.

\subsubsection{LEC compression}
The LEC algorithm is described in detailed pseudo code in \cite{Marcelloni:2009ja}. Our implementation follows this pseudo code as closely as possible, and is listed in Appendix \ref{app-lec-source}. The input is a set of 256 16-bit ECG measurements downloaded from PhysioNet \cite{physionet-ecg-data}.

\subsubsection{MoteTrack}
\label{sec-evaluation-benchmark-implementation-motetrack}
The \mybench{MoteTrack} application uses received signal strength (RSSI) measurements from a number of beacon nodes to do indoor localisation. It contains a database in flash memory of reference RSSI signatures, stored in a complex structure of many small structs and arrays in C.

The memory overhead when translating this directly to Java was too high to run the application, forcing us to make two modifications. First, the original source has the option to list RSSI signatures at different transmission powers, but the authors note this may not always improve results. The original C code only uses a single transmission power setting, which results in arrays of a single element that get optimised away at compile time. In the Java version there were replaced with simple variables. Second, a two element array with RSSI values for different channels was flattened into two separated variables to eliminate the overhead from allocating too many small arrays.

Again, the constant array optimisation was used to place the data in flash memory. Since this only allows arrays of integer types, the single array of C nested structs was split into 7 arrays for the individual fields. Without this optimisation, it would be impossible to implement this application in Java because the 20 KB database is too large to fit in RAM.

Thus, while our Java implementation of \mybench{MoteTrack} does execute the same algorithm as the C version, we were force to modify its implementation significantly, which clearly highlights some of the weaknesses of current sensor node VMs. These changes do not affect the results of the current version of the code, but we note that while it would be possible to use multiple transmission powers or more channels in the C version, this would require too much memory for the Java version.

\subsubsection{Heat detection}
The \mybench{heat detection} application is adapted from code used by a different project in our group to track persons and fire hazards using Raspberry Pi devices equipped with an 8x8 heat sensor.

It contains two phases: first the heat sensor is calibrated with no heat sources in view to determine the average and standard deviation of the sensor readings. In the next phase the algorithm tracks the position of a person moving within the field of view of the sensor, and detects extreme temperatures that may indicate a fire. In the evaluation both phases are listed separately.

The calibration phase was modified to allow it to run on the more resource-constrained sensor node, but the results of the calibration are identical. The code for the detection phase was copied directly from the source used on the Raspberry Pi, but modified slightly to avoid repeatedly allocating temporary objects as described in Section \ref{sec-no-gc}.

Our implementation reads sensor measurements using a native call to read from a table in flash memory, simulating to how a real version would use a library call to read from a sensor.

The two phases are listed separately in the evaluation.


\subsection{Experimental setup}
Each benchmark is implemented as a C and a Java version. We compile these using \mycode{javac} version 1.8.0, ProGuard 5.2.1, and \mycode{avr-gcc} version 4.9.1. The C benchmarks are compiled at optimisation level -O3, the rest of the VM at -Os.

A manual examination of the compiled code produced by \mycode{avr-gcc} revealed some points where more efficient code could have been generated. But with the exception of the lack of some constant shift optimisations discussed in Section \ref{sec-opt-constant-shift}, these did not affect performance by more than a few percent. This leads us to believe \mycode{avr-gcc} is a fair benchmark to compare to.

Each benchmark was run in the cycle-accurate Avrora simulator \cite{Titzer:2005vb}, emulating the ATmega128 processor. Avrora was modified to emit traces of the AOT translation process and of the run-time performance. During AOT translation, the VM writes to a specific memory address monitored by Avrora to inform it of each step in the process. When running both the C and AOT compiled benchmarks, Avrora tracks the number of cycles spent in each instruction. These traces, combined with debug output from the infuser and disassembled native code provide a detailed view of the performance on a per-instruction basis.

The main measure for both code size and performance is the overhead compared to optimised native C. To compare different benchmarks, this overhead is normalised to a percentage of the number of bytes or cpu cycles used by the native implementation: a 100\% overhead means the AOT compiled version takes twice as long to run, or twice as many bytes to store. The exact results can vary depending on factors such as which benchmarks are chosen, the input data, etc., but the general trends are all quite stable.
