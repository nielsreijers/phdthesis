\section{Constant arrays}
\label{sec-evaluation-const-array}
\begin{table}
\centering
\caption{Effect of constant arrays on size and performance}
\label{tbl-const-array}
	\scriptsize
    \centerline{
    \begin{tabular}{lrrrrrrrr} % UPDATED 20180327
    \toprule
                                & \multicolumn{2}{c}{RC5}   & \multicolumn{2}{c}{FFT}   & \multicolumn{2}{c}{LEC} & \multicolumn{2}{c}{MoteTrack} \\
Size of constant data           & \multicolumn{2}{c}{200}   & \multicolumn{2}{c}{2,048} & \multicolumn{2}{c}{51}  & \multicolumn{2}{c}{20,560}    \\
    \midrule
    \midrule
Using constant arrays           &       no &            yes &       no &            yes &        no &         yes &          no &             yes \\
Performance overhead            &   19.5\% &         19.5\% &   17.7\% &         17.7\% &    86.5\% &      84.6\% &  cannot run &         156.3\% \\
% Size in flash for @ConstArray version: 4 byte overhead per array for string table forward pointers and length fields
% Size in flash for _nca version       : the AOT size of the <clinit> taken from stdoutlog.txt
Size of constant data in flash  &     1998 &            204 &   26,714 &          2,052 &       930 &          59 &  cannot run &          20,588 \\
% 8 byte overhead per array for 5 byte heap header and 3 byte array header. MoteTrack would use 20,616 bytes if is could run.
Size of constant data in RAM    &      208 &              0 &    2,056 &              0 &        67 &           0 &  cannot run &               0 \\
    \bottomrule
    \end{tabular}
    }
\end{table}


Four benchmarks contain arrays of constant data, which were stored in flash memory by placing them in classes with the \mycode{@ConstArray} annotation. To evaluate the effect of this optimisation, we compare them to versions without this annotation. The results are shown in Table \ref{tbl-const-array}. There are three advantages to this optimisation: a small improvement in performance, reduced code size, and reduced memory usage.

When using constant arrays, the id of the array to read from is a bytecode parameter in the \mycode{GETCONSTARRAY} instruction. No reference to the array needs to be loaded on the stack, and the calculation to find the address of the target element is slightly easier, which results in a modest reduction in performance overhead of 1.9\% for the \mybench{LEC} benchmark.

The real advantage however, is the reduction in code size and memory usage. Without this optimisation, an array of constant data is transformed into normal bytecode that will create an array object on the heap and fill each element individually, as shown in Listing \ref{lst-constant-array-initialisation}.

The class initialiser uses four bytecode instructions per element to fill each element of the array. For an array of bytes, this can take up to 7 bytes of bytecode for each byte of data, which increases even further after AOT compilation. In the \mybench{LEC} benchmark this results in a class initialiser of 930 bytes, over \emph{18 times} the size of the original data. 

For such a small array this might still be acceptable, but the 26 KB needed to store \mybench{FFT}'s 2 KB of data is a significant overhead, and while \mybench{MoteTrack}'s 20 KB of data could fit in flash memory, the resulting class initialiser cannot. When using the constant array optimisation, the array is stored as raw data in the constant pool, resulting in just 4 bytes of overhead per array.

The final, and most significant advantage of this optimisation is that the array is no longer stored in RAM. Again, the 67 bytes of RAM needed to store \mybench{LEC}'s two constant arrays, each with 8 bytes overhead for the heap and array headers, may be acceptable. For \mybench{RC5} the 208 byte RAM overhead is starting to be significant, and while the \mybench{FFT} benchmark can still run without the constant array optimisation, its array consumes over half of the ATmega128's 4 KB of RAM. For \mybench{MoteTrack}, the size of its constant arrays is well over the amount of RAM available, making it impossible to run this benchmark without the constant array optimisation.